\section{Manual de uso}
En esta secci\'on veremos los requerimientos para correr el programa y c\'omo utilizarlo. Los requerimientos son:

\begin{itemize}
  \item Python 2.7
  \item ply-3.8
  \item Alg\'un visor de SVG para poder visualizar el output
\end{itemize}

El programa se puede correr de distintas formas:

\begin{itemize}
  \item {
  	\par Completamente interactivo:
  	\par Tanto el input como el output es dentro del programa. Para ello basta con ejecutarlo del siguiente modo: \texttt{python main.py}
  	}
  \item {
  	\par Input por l\'inea de comando:
  	\par La f\'ormula se introducir\'a por linea de comandos pero el resultado se ver\'a por pantalla.
  	\par Para esto ejecutar: \texttt{python main.py $--$formula "FORMULA"} donde FORMULA es la f\'ormula que se desee convertir. Las comillas \textbf{no} son opcionales ya que el m\'odulo de Python con el que parametrizamos la entrada parece no comprender el s\'imbolo de los super\'indices.
  }
  \item {
  	\par Output por file: 
  	\par La f\'ormula se introduce dentro del programa pero el output se especifica por l\'inea de comandos: \textsc{python main.py $--$output FILENAME} 
  	\par Ejemplo: \texttt{python main.py $--$output salida.svg}
  }
  \item {
  	\par L\'inea de comandos: 
  	\par El input y el output se especifica en el comando para esto: \texttt{python main.py $--$formula "FORMULA" $--$output FILENAME}
  }
\end{itemize}
