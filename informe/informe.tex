\documentclass[10pt, a4paper,english,spanish]{article}

% \usepackage{a4wide}
\parindent = 0 pt
\parskip = 11 pt
\usepackage[width=15.5cm, left=3cm, top=2.5cm, height= 24.5cm]{geometry}
%Margenes de la pagina.  otra opcion, usar \usepackage{a4wide}
%\usepackage[paper=a4paper, left=0.8cm, right=0.8cm, bottom=1.3cm, top=0.9cm]{geometry}
\usepackage{color}
\usepackage{amsmath}
\usepackage{amsfonts}
%este paquete permitcodebe incluir acentos.  Notar que espera un formato ANSI-blah de archivo.  Si en lugar de eso se tiene un utf8 (usual en los linux), entonces usar \usepackage[utf8]{inputenc}
\usepackage[utf8]{inputenc}

%Este paquete es para que algunos titulos (como Tabla de Contenidos) esten en castellano
\usepackage[spanish]{babel}
\usepackage{csquotes}
%El siguiente paquete permite escribir la caratula facilmente
\usepackage{caratula}

\usepackage{framed}

\usepackage{graphicx}
\usepackage{float}

\usepackage{algorithm}
\usepackage{algorithmic}

%\usepackage{algpseudocode}

\newcommand{\real}{\mathbb{R}}
\newcommand{\nat}{\mathbb{N}}

\newcommand{\revJ}[1]{{\color{red} #1}}

%Datos para la caratula
\materia{Teoría de Lenguajes}

\titulo{Trabajo Pr\'actico}

\integrante{Claverino, Daniel}{273/10}{dclave@gmail.com}
\integrante{Conde, Fernando}{423/09}{ferconde87@hotmail.com}
\integrante{Forte, Martín}{363/10}{martinforte@yahoo.com.ar}

\begin{document}

%esto construye la caractula
\maketitle 

\tableofcontents
  \newpage

\section{Introducci\'on}
El objetivo de este trabajo pr\'actico es desarrollar un compositor de f\'ormulas matem\'aticas. El mismo tomar\'a como entrada la descripci\'on de una f\'ormula en una versi\'on muy simplificada del lenguaje utilizado por \LaTeX \hspace{0.1cm}y producir\'a como salida un archivo SVG (Scalable Vector Graphics).
  \newpage
%\input{desarrollo.tex}
%  \newpage
\section{Gramática}
La gramática se nos fue presentada fue la siguiente:\\

\begin{tabular}{ l c l }
    E & $\rightarrow$ & EE \\
    & $|$ & E$\wedge$E \\
    & $|$ & E\_E \\
    & $|$ & E$\wedge$E\_E \\
    & $|$ & E\_E$\wedge$E \\
    & $|$ & E$\slash$E \\
    & $|$ & (E) \\
    & $|$ & \{E\} \\
    & $|$ & l \\
\end{tabular}

Dado que dicha gramática tenia problemas de ambiguedad, los cuales preferimos evitar y que había ciertas restricciones del lenguaje que no estaban contempladas decidimos modificarla.
A continuación presentamos la gramática y la explicación de porque representa el mismo lenguaje: \\

Ahora el símbolo distiguido es $S$ el cuál deriva en $E$. Uno de los problemas que teníamos con la ambiguedad es que $E$ podía
ser cualquier tipo de expresión con lo cual decidimos darle una cierta jerarquía a cada expresión (aprovechando que la gramática lo solicitaba) y de este modo
deshacernos de la ambiguedad.\\


\begin{tabular}{ l c l }
    S &        $\rightarrow$ & E \\
    E &        $\rightarrow$ & E $\slash$ CONCAT $|$ CONCAT \\
    CONCAT &   $\rightarrow$ & CONCAT ELEMENTS $|$ ELEMENTS \\
    ELEMENTS & $\rightarrow$ & INDEXES $|$ NOINDEX \\
\end{tabular} \\
$E$ puede convertirse en divisiones o bien pasar a ser una $CONCAT$enación recursiva de $ELEMENTS$ donde cada elemento de la concatenación
puede ser una expresión que contiene $INDEXES$ o bien $NOINDEXES$.\\

$INDEXES$, como su nombre lo indica, contiene todas las expresiones que tienen índices y cada expresión no puede
estar formada por expresiones $INDEXES$ ya que los indices no son asociativos(por enunciado) y porque la concatenación
de subíndices con superíndices genera ambiguedad con la expresión original E $\rightarrow$ E\_E$\wedge$E. \\


\begin{tabular}{ l c l }
    INDEXES &  $\rightarrow$ & SUPER $|$ SUB $|$ SUPSUB $|$ SUBSUP \\
    SUPER &    $\rightarrow$ & NOINDEX$\wedge$NOINDEX \\
    SUB &      $\rightarrow$ & NOINDEX\_NOINDEX \\
    SUPSUB &   $\rightarrow$ & NOINDEX$\wedge$NOINDEX\_NOINDEX \\
    SUBSUP &   $\rightarrow$ & NOINDEX\_NOINDEX$\wedge$NOINDEX \\
\end{tabular}\\

$NOINDEX$ son todas aquellas expresiones que no son formadas por índices: $PAR$, $GROUP$ y $ID$. \\


\begin{tabular}{ l c l }
    NOINDEX &  $\rightarrow$ & PAR $|$ GROUP $|$ ID \\    
    PAR &      $\rightarrow$ & (E) \\
    GROUP &    $\rightarrow$ & \{E\} \\
    ID &       $\rightarrow$ & l \\
\end{tabular} \\

\newpage
\section{Manual de uso}
En esta sección veremos los requerimientos para correr el programa y como utilizarlo. Los requerimientos son:



\newpage
\section{C\'odigo}

\par En \textbf{main.py} usando \textit{ply} tenemos definidos el \textit{lexer} y el \textit{parser}, una funcion encargada de los tests y el programa propiamente dicho. El Lexer no merece mucha atenci\'on, as\'i que pasaremos directamente al parser:
\par Como ply utiliza LALR y \'este s\'olo permite atributos sintetizados, decidimos tener un \'unico atributo que va formando un \'arbol a medida que se hacen las reducciones. Dicho \'arbol no refleja la sintaxis, sino que es m\'as bien una representaci\'on de las posiciones y tama\~nos de los caracteres de la f\'ormula, junto con su dependencia.
\par En \textbf{Node.py} est\'a el c\'odigo de los Nodos de este \'arbol.
\par Tomando como ejemplo la producci\'on $CONCAT_1 \rightarrow CONCAT_2$ $ELEMENTS$, donde en ply se separan los terminales y no-terminales como $p[0]$ $(CONCAT_1)$, $p[1]$ $(CONCAT_2)$ y $p[2]$ $(ELEMENTS)$, lo que hacemos es $p[0] = ConcatNode(p[1], p[2])$. Es decir, representamos la concatenaci\'on de los dos no-terminales ya procesados como \'arboles en $p[1]$ y $p[2]$.
\par Por otro lado, en $GROUP \rightarrow \{ E \}$, tenemos $p[0]$ $(GROUP)$, $p[1]$ $(\{)$, $p[2]$ $(E)$ y $p[3]$ $(\})$, y en este caso pasamos directamente el \'arbol procesado en $p[2]$. Es decir, $p[0] = p[2]$, ya que no hay ning\'un cambio respecto a la posicion o tama\~no.
\par Estos dos ejemplos muestran que el \'arbol que generamos tiende a ser sint\'actico pero no lo es necesariamente. De hecho, definimos un \texttt{LineNode} que representa la l\'inea de divisi\'on, y que se instancia en la implementaci\'on de \texttt{DivideNode}.
\par Finalmente, en la producci\'on $S \rightarrow E$ simplemente completamos el \'arbol con la ra\'iz y llamamos al m\'etodo $toSvg()$ de \'esta. $p[0] = MainNode(p[1]).toSvg()$, donde $MainNode$ representa la ra\'iz del \'arbol y $toSvg()$ va generando y concatenando los $toSvg$ de los subnodos para obtener el svg final de la f\'ormula.

\par \phantom{a}
\par A continuaci\'on inclu\'imos unos ejemplos de resultados obtenidos y luego presentamos el c\'odigo de \textbf{main.py} y \textbf{Node.py}:

\newpage
\subsubsection{Ejemplos}
\begin{figure}[htbp]
	\centering
	\begin{subfigure}[b]{0.3\textwidth}
	  \includegraphics[width=\textwidth]{imgs/test1}
	  \caption{$(a \_ 5-c/b-1)-c$}
	\end{subfigure}
	\quad
	\begin{subfigure}[b]{0.3\textwidth}
	  \includegraphics[width=\textwidth]{imgs/test2}
	  \caption{$\{a \wedge \{5 \wedge 6\}-c \_ \{k \wedge 9\}/b \_ i\}-c$}
	\end{subfigure}
	\quad
	\begin{subfigure}[b]{0.3\textwidth}
	  \includegraphics[width=\textwidth]{imgs/test3}
	  \caption{$(A \wedge BC \wedge D/E \wedge F \_ G+H)-I$}
	\end{subfigure}
	
	\hfill
	
	\hfill
	
	\hfill
	
	\begin{subfigure}[b]{\textwidth}
	  \includegraphics[width=\textwidth]{imgs/test4}
	  \caption{$A+(B)\{G \wedge \{(F \wedge e \_ E/(2))\}-(Q \_ \{E \_ \{\{5\}+E \_ \{E \_ \{E \_ D\}\}\}\}-Y)+X \wedge K \_ J/Y\}-\{(80)/(2)\}-\{C \wedge \{G \wedge \{G \wedge \{G\}\}\}/5\}/(\{8+4+7\}+5/ee) \wedge \{-i\}$}
	\end{subfigure}
\end{figure}

\newpage
\subsection{main.py}
\lstinputlisting{../src/main.py}

\newpage
\subsection{Node.py}
\lstinputlisting{../src/Node.py}
%  \newpage
%\input{apendices.tex}
%  \newpage
\section{Conclusiones}
Una de las cosas que notamos es que hay gram\'aticas que son ambiguas y eso las hace f\'cilmente comprendibles
por humanos pero las computadoras \textquote{prefieren} gram\'aticas que no lo sean. Aprendimos que con un poco de trabajo
podemos, no solo desambiguar una gram\'tica, sino tambi\'en modificarla para que algunas restricciones que estaban dadas sem\'anticamente
queden abarcadas por la gram\'atica como lo fueron la no asociatividad de los exponentes y la precedencia de algunos operadores. \\

Pudimos ver de una forma pr\'actica como el parser interact\'ua con el lexer y sem\'antica para formar cosas muy poderosas de una
manera muy simple, ya que con peque\~nos ladrillitos como lo son las producciones se generan de forma recursiva cosas inimaginables.
Despu\'es de este trabajo pr\'actico, ver cosas como las que hace \LaTeX nos dejaron de parecer \textquote{magia negra} y pudimos entender
que el concepto por el cual se rigen es el mismo. \\
 % \newpage

\end{document}