\documentclass[10pt, a4paper,english,spanish]{article}

% \usepackage{a4wide}
\parindent = 0 pt
\parskip = 11 pt
\usepackage[width=15.5cm, left=3cm, top=2.5cm, height= 24.5cm]{geometry}
%Margenes de la pagina.  otra opcion, usar \usepackage{a4wide}
%\usepackage[paper=a4paper, left=0.8cm, right=0.8cm, bottom=1.3cm, top=0.9cm]{geometry}
\usepackage{color}
\usepackage{amsmath}
\usepackage{amsfonts}
%este paquete permitcodebe incluir acentos.  Notar que espera un formato ANSI-blah de archivo.  Si en lugar de eso se tiene un utf8 (usual en los linux), entonces usar \usepackage[utf8]{inputenc}
\usepackage[utf8]{inputenc}

%Este paquete es para que algunos titulos (como Tabla de Contenidos) esten en castellano
\usepackage[spanish]{babel}

%El siguiente paquete permite escribir la caratula facilmente
\usepackage{caratula}

\usepackage{framed}

\usepackage{graphicx}
\usepackage{float}

\usepackage{algorithm}
\usepackage{algorithmic}

%\usepackage{algpseudocode}

\newcommand{\real}{\mathbb{R}}
\newcommand{\nat}{\mathbb{N}}

\newcommand{\revJ}[1]{{\color{red} #1}}

%Datos para la caratula
\materia{Teoría de Lenguajes}

\titulo{Trabajo Pr\'actico}

\integrante{Claverino, Daniel}{273/10}{dclave@gmail.com}
\integrante{Conde, Fernando}{423/09}{ferconde87@hotmail.com}
\integrante{Forte, Martín}{363/10}{martinforte@yahoo.com.ar}

\begin{document}

%esto construye la caractula
\maketitle 

\tableofcontents
  \newpage

\section{Introduccion}
El objetivo de este trabajo pr\'actico es desarrollar un compositor de f\'ormulas matem\'aticas. El mismo tomar\'a como entrada la descripci\'on de una f\'ormula en una versi\'on muy simplificada del lenguaje utilizado por \LaTeX \hspace{0.1cm}y producir\'a como salida un archivo SVG (Scalable Vector Graphics).
%  \newpage
%\input{desarrollo.tex}
%  \newpage
%\input{resultados.tex}
%  \newpage
%\input{conclusiones.tex}
%  \newpage
%\input{apendices.tex}
%  \newpage
%\input{referencias.tex}
 % \newpage

\end{document}