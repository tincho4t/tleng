\documentclass[10pt, a4paper,english,spanish]{article}

% \usepackage{a4wide}
\parindent = 0 pt
\parskip = 11 pt
\usepackage[width=15.5cm, left=3cm, top=2.5cm, height= 24.5cm]{geometry}
%Margenes de la pagina.  otra opcion, usar \usepackage{a4wide}
%\usepackage[paper=a4paper, left=0.8cm, right=0.8cm, bottom=1.3cm, top=0.9cm]{geometry}
\usepackage{color}
\usepackage{amsmath}
\usepackage{amsfonts}
%este paquete permitcodebe incluir acentos.  Notar que espera un formato ANSI-blah de archivo.  Si en lugar de eso se tiene un utf8 (usual en los linux), entonces usar \usepackage[utf8]{inputenc}
\usepackage[utf8]{inputenc}

%Este paquete es para que algunos titulos (como Tabla de Contenidos) esten en castellano
\usepackage[spanish]{babel}

%El siguiente paquete permite escribir la caratula facilmente
\usepackage{caratula}

\usepackage{framed}

\usepackage{graphicx}
\usepackage{float}

\usepackage{algorithm}
\usepackage{algorithmic}

%\usepackage{algpseudocode}

\newcommand{\real}{\mathbb{R}}
\newcommand{\nat}{\mathbb{N}}

\newcommand{\revJ}[1]{{\color{red} #1}}

%Datos para la caratula
\materia{Teoría de Lenguajes}

\titulo{Trabajo Pr\'actico}

\integrante{Claverino, Daniel}{273/10}{dclave@gmail.com}
\integrante{Conde, Fernando}{423/09}{ferconde87@hotmail.com}
\integrante{Forte, Martín}{363/10}{martinforte@yahoo.com.ar}

\begin{document}

%esto construye la caractula
\maketitle 

\tableofcontents
  \newpage

\section{Introduccion}
El objetivo de este trabajo pr\'actico es desarrollar un compositor de f\'ormulas matem\'aticas. El mismo tomar\'a como entrada la descripci\'on de una f\'ormula en una versi\'on muy simplificada del lenguaje utilizado por \LaTeX \hspace{0.1cm}y producir\'a como salida un archivo SVG (Scalable Vector Graphics).
  \newpage
%\input{desarrollo.tex}
%  \newpage
\section{Gramática}
Las producciones de la gram\'atica que se nos fue presentada son de la forma:\\

\begin{tabular}{ l c l }
    E & $\rightarrow$ & EE \\
    & $|$ & E$\wedge$E \\
    & $|$ & E\_E \\
    & $|$ & E$\wedge$E\_E \\
    & $|$ & E\_E$\wedge$E \\
    & $|$ & E$\slash$E \\
    & $|$ & (E) \\
    & $|$ & \{E\} \\
    & $|$ & l \\
\end{tabular}

Como dicha gram\'atica tiene problemas de ambig\"uedad y hay ciertas restricciones del lenguaje que no est\'an contempladas, decidimos modificarla.
A continuaci\'on presentamos nuestra gram\'atica y la explicaci\'on de por qu\'e representa el mismo lenguaje: \\

El s\'imbolo distiguido ahora es $S$, el cual deriva en $E$, y nos permite saber cu\'ando se reduce a la ra\'iz. Uno de los problemas que ten\'iamos con la ambig\"uedad es que $E$ pod\'ia ser cualquier tipo de expresi\'on, con lo cual decidimos darle una cierta jerarqu\'ia a cada expresi\'on (aprovechando que la gram\'atica lo solicitaba) y de este modo deshacernos de la ambig\"uedad.\\


\begin{tabular}{ l c l }
    S &        $\rightarrow$ & E \\
    E &        $\rightarrow$ & E $\slash$ CONCAT $|$ CONCAT \\
    CONCAT &   $\rightarrow$ & CONCAT ELEMENTS $|$ ELEMENTS \\
    ELEMENTS & $\rightarrow$ & INDEXES $|$ NOINDEX \\
\end{tabular} \\
$E$ puede convertirse en divisiones o bien pasar a ser una $CONCAT$enaci\'on recursiva de $ELEMENTS$ donde cada elemento de la concatenaci\'on
puede ser una expresi\'on que contiene $INDEXES$ o bien $NOINDEXES$.\\

$INDEXES$, como su nombre lo indica, contiene todas las expresiones que tienen \'indices. No pueden estar directamente formadas por expresiones $INDEXES$ ya que los indices no son asociativos (por enunciado) y porque la concatenaci\'on
de sub\'indices con super\'indices generar\'ia ambig\"uedad, por ejemplo con la expresi\'on original E $\rightarrow$ E\_E$\wedge$E. \\


\begin{tabular}{ l c l }
    INDEXES &  $\rightarrow$ & SUPER $|$ SUB $|$ SUPSUB $|$ SUBSUP \\
    SUPER &    $\rightarrow$ & NOINDEX$\wedge$NOINDEX \\
    SUB &      $\rightarrow$ & NOINDEX\_NOINDEX \\
    SUPSUB &   $\rightarrow$ & NOINDEX$\wedge$NOINDEX\_NOINDEX \\
    SUBSUP &   $\rightarrow$ & NOINDEX\_NOINDEX$\wedge$NOINDEX \\
\end{tabular}\\

$NOINDEX$ son todas aquellas expresiones que no son formadas por \'indices: $ID$, $PAR$ y $GROUP$. \\


\begin{tabular}{ l c l }
    NOINDEX &  $\rightarrow$ & PAR $|$ GROUP $|$ ID \\    
    PAR &      $\rightarrow$ & (E) \\
    GROUP &    $\rightarrow$ & \{E\} \\
    ID &       $\rightarrow$ & l \\
\end{tabular} \\

\newpage
\section{Manual de uso}
En esta secci\'on veremos los requerimientos para correr el programa y c\'omo utilizarlo. Los requerimientos son:

\begin{itemize}
  \item Python 2.7
  \item ply-3.8
  \item Alg\'un visor de SVG para poder visualizar el output
\end{itemize}

El programa se puede correr de distintas formas:

\begin{itemize}
  \item {
  	\par Completamente interactivo:
  	\par Tanto el input como el output es dentro del programa. Para ello basta con ejecutarlo del siguiente modo: \texttt{python main.py}
  	}
  \item {
  	\par Input por l\'inea de comando:
  	\par La f\'ormula se introducir\'a por linea de comandos pero el resultado se ver\'a por pantalla.
  	\par Para esto ejecutar: \texttt{python main.py $--$formula ``FORMULA''} donde FORMULA es la f\'ormula que se desee convertir. Las comillas son opcionales a menos que hayan espacios en la f\'ormula.
  }
  \item {
  	\par Output por file: 
  	\par La f\'ormula se introduce dentro del programa pero el output se especifica por linea de comandos: \textsc{python main.py $--$output FILENAME} 
  	\par Ejemplo: \texttt{python main.py $--$output salida.svg}
  }
  \item {
  	\par L\'inea de comandos: 
  	\par El input y el output se especifica en el comando para esto: \texttt{python main.py $--$formula ``FORMULA'' $--$output FILENAME}
  }
\end{itemize}

%  \newpage
%\input{apendices.tex}
%  \newpage
%\input{referencias.tex}
 % \newpage

\end{document}