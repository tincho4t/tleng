\section{Código}

En \textbf{main.py} usando \textit{ply} tenemos definidos el \textit{lexer} y el \textit{parser}, mientras que en \textbf{Node.py} está la implementación de un árbol que se va generando a media que se parsea la cadena de entrada.

\newpage
\textbf{main.py:}
\begin{verbatim}

import argparse

from Node import MainNode
from Node import DivideNode
from Node import ConcatNode
from Node import SuperIndexNode
from Node import SubIndexNode
from Node import SuperSubIndexNode
from Node import CharacterNode
from Node import ParenthesisNode

tokens = (
    'SUB',
    'SUPER',
    'DIVIDE',
    'LPAREN',
    'RPAREN',
    'LLLAVE',
    'RLLAVE',
    'CARACTERES'
)
# Tokens

t_SUB   = r'_'
t_SUPER   = r'\^'
t_DIVIDE  = r'/'
t_LPAREN  = r'\('
t_RPAREN  = r'\)'
t_LLLAVE  = r'\{'
t_RLLAVE  = r'\}'
t_CARACTERES = r'[^_\^/\(\)\{\}]'



# Ignored characters
t_ignore = " \t"

def t_newline(t):
    r'\n+'
    t.lexer.lineno += t.value.count("\n")

def t_error(t):
    print("Illegal character '%s'" % t.value[0])
    t.lexer.skip(1)

# Build the lexer
import ply.lex as lex
lex.lex()



precedence = (
    ('left','SUB','SUPER'),
    ('left','DIVIDE'),
)


# S -> E
def p_statement_expression(p):
    'statement : expression'
    p[0] = MainNode(p[1]).toSvg()

# E -> E / CONCAT
def p_expression_divide(p):
    'expression : expression DIVIDE concat'
    p[0] = DivideNode(p[1], p[3])

# E -> CONCAT
def p_expression_concat(p):
    'expression : concat'
    p[0] = p[1]

# CONCAT -> CONCAT ELEMENTS
def p_concat_concat(p):
    'concat : concat elements'
    p[0] = ConcatNode(p[1], p[2])

# CONCAT -> ELEMENTS
def p_concat_elements(p):
    'concat : elements'
    p[0] = p[1]

# ELEMENTS -> INDEXES
def p_elements_indexes(p):
    'elements : indexes'
    p[0] = p[1]

# ELEMENTS -> NOINDEX
def p_elements_noindex(p):
    'elements : noindex'
    p[0] = p[1]

# SUPER -> NOINDEX^NOINDEX
def p_super(p):
    'super : noindex SUPER noindex'
    p[0] = SuperIndexNode(p[1], p[3])

# SUB -> NOINDEX_NOINDEX
def p_sub(p):
    'sub : noindex SUB noindex'
    p[0] = SubIndexNode(p[1], p[3])

# SUPSUB -> NOINDEX^NOINDEX_NOINDEX
def p_superindex_subindex(p):
    'supsub : noindex SUPER noindex SUB noindex'
    p[0] = SuperSubIndexNode(p[1], p[3], p[5])

# SUBSUP -> NOINDEX_NOINDEX^NOINDEX
def p_subindex_superindex(p):
    'subsup : noindex SUB noindex SUPER noindex'
    p[0] = SuperSubIndexNode(p[1], p[5], p[3])

# INDEXES -> SUPER | SUB | SUPSUB | SUBSUP
def p_indexes(p):
    '''indexes : super
               | sub
               | supsub
               | subsup'''
    p[0] = p[1]

#NOINDEX -> PAR | GROUP | ID
def p_noindex(p):
    '''noindex : parenthesis
               | group
               | id'''
    p[0] = p[1]

# ID -> l
def p_id(p):
    'id : CARACTERES'
    p[0] = CharacterNode(p[1])

# PAR -> (E)
def p_parenthesis(p):
    'parenthesis : LPAREN expression RPAREN'
    p[0] = ParenthesisNode(p[2])

# GROUP -> {E}
def p_group(p):
    'group : LLLAVE expression RLLAVE'
    p[0] = p[2]

def p_error(p):
    if(p):
        if(s): # Si esta seteada la variable global la uso para expresar mejor el error
            if(len(s) > p.lineno+1):
                error = "Error en el caracter '%s'. Contexto: '%s' '%s' '%s'" % 
                		(p.value, s[0:p.lineno], s[p.lineno], s[p.lineno+1:])
            else:
                error = "Error en el caracter '%s'. Contexto: '%s'" % (p.value, s[0:p.lineno])
        else:
            error = "Error en el caracter '%s' en la posicion %d." % (p.value, p.lineno)
    else: # en algunos casos p no viene definido y no hay mucha mas informacion para mostrar
        error = "Error de sintaxis."
    raise SyntaxError(error)

import ply.yacc as yacc
yacc.yacc()

def test():
    # Casos que tiene que tiene que reconocer
    assert test_accpet('(a_5-c/b-1)-c')
    assert test_accpet('{a^5-c/b}-c')
    assert test_accpet('{a^{5^6}-c_{{k^9}}/b_i}-c')
    assert test_accpet('(10+5/2)')
    assert test_accpet('1_2^{3_4^{5_6^7}}')
    assert test_accpet('(A^BC^D/E^F_G+H)-I')
    assert test_accpet('A+(B){G^{(F^e_E/(2))}-(Q_{E_{{5}+E_{E_{E_D}}}}-Y)+X^K_J/Y}-
    					{(80)/(2)}-{C^{G^{G^{G}}}/5}/({8+4+7}+5/ee)^{-i}')
    assert test_not_accept('1^2^3')
    assert test_not_accept('1_2_3')
    assert test_not_accept('_')
    assert test_not_accept('_1')
    assert test_not_accept('^')
    assert test_not_accept('^1')
    assert test_not_accept('1_')
    assert test_not_accept('1/')
    assert test_not_accept('1/^')
    assert test_not_accept('{1+2')
    assert test_not_accept('1+2}')
    assert test_not_accept('(1+2')
    assert test_not_accept('3(1+2')
    assert test_not_accept('1+2)')
    assert test_not_accept('()')
    assert test_not_accept('())')
    assert test_not_accept('{{1}')
		
def test_not_accept(s):
    success = True
    try:
        s = yacc.parse(s)
        success = False # Si no falla paso por aca
    except SyntaxError:
        pass
    return success


# Si no pincha es porque se parseo correctamente
def test_accpet(s):
    yacc.parse(s)
    return True

s = None # Setea la variable global para poder mostrar mas informacion del error de parseo
test()

# Parseo de la entrada
parser = argparse.ArgumentParser(description='Conversor de formulas a SVG.')
parser.add_argument('--formula', type=str, help='Formula en formato pseudo latex')
parser.add_argument('--output', type=str, help='nombre del archivo donde se guardara 
					el resultado')
args = parser.parse_args()
if(args.formula):
    s = args.formula
else:
    try:
        s = raw_input('Formula :> ')
    except EOFError:
        pass
try:
    bald = yacc.parse(s)
except SyntaxError as e:
    print(e) # Si hay un error de parseo lo muestro por pantalla y termino
    raise SystemExit

if(args.output):
    with open(args.output, 'w') as f:
    	f.write(bald)
else:
    print(bald)
\end{verbatim}

\newpage

\textbf{Node.py:}
\begin{verbatim}
ONT_SIZE = 48
CHAR_WIDTH = .6 * FONT_SIZE
CHAR_HEIGHT = 1.0 * FONT_SIZE
CHAR_UPPER_HEIGHT = .8 * FONT_SIZE
CHAR_LOWER_HEIGHT = .2 * FONT_SIZE
PARENTHESIS_SPACING = .5 * FONT_SIZE
LINE_SPACING = .1 * FONT_SIZE
SUPER_SPACING = -0.5 * CHAR_UPPER_HEIGHT
SUB_SPACING = -0.5 * CHAR_UPPER_HEIGHT

class Node(object):
    def __init__(self, wid, hlw, hup):
        self.x = 0
        self.y = 0
        self.wid = wid
        self.hlw = hlw
        self.hup = hup
        self.scale = FONT_SIZE
        self.subNodes = []

    def getPosition(self):
        return (self.x, self.y)

    def getHeights(self):
        return (self.hlw, self.hup)

    def getLowerHeight(self):
        return self.hlw

    def getUpperHeight(self):
        return self.hup

    def getHeight(self):
        return self.hup + self.hlw

    def getWidth(self):
        return self.wid

    def getScale(self):
        return self.scale

    def setPosition(self, position):
        nx, ny = position
        dx = nx - self.x
        dy = ny - self.y
        self.movePosition(dx, dy)

    def scaleBy(self, scale):
        self.scale *= scale
        self.x *= scale
        self.y *= scale
        self.wid *= scale
        self.hup *= scale
        self.hlw *= scale
        for node in self.subNodes:
            node.scaleBy(scale)

    def movePosition(self, x, y):
        self.x += x
        self.y += y
        for node in self.subNodes:
            node.movePosition(x, y)

    def addNode(self, node):
        self.subNodes.append(node)

    def toSvg(self):
        return "".join([node.toSvg() for node in self.subNodes])

class MainNode(Node):
    def __init__(self, node):
        wid = node.getWidth()
        hlow, hupp = node.getHeights()
        super(MainNode, self).__init__(wid, hlow, hupp)
        self.addNode(node)

    def toSvg(self):
        return '''<?xml version="1.0" standalone="no"?>
<!DOCTYPE svg PUBLIC "-//W3C//DTD SVG 1.1//EN"
"http://www.w3.org/Graphics/SVG/1.1/DTD/svg11.dtd">
<svg xmlns="http://www.w3.org/2000/svg" version="1.1">
<g transform="translate(10, %.2f) " font-family="Courier">\n''' % (10+ self.hup - self.y) + \
            super(MainNode, self).toSvg() + \
            '''</g></svg>'''

class CharacterNode(Node):
    def __init__(self, char):
        super(CharacterNode, self).__init__(CHAR_WIDTH, CHAR_LOWER_HEIGHT, CHAR_UPPER_HEIGHT)
        self.char = char

    def toSvg(self):
        x, y = self.getPosition()
        return "<text x='%.2f' y='%.2f' font-size='%.2f'>%s</text>\n" 
        		% (x, y, self.scale, self.char)

class ConcatNode(Node):
    def __init__(self, nodeA, nodeB):
        wid = nodeA.getWidth() + nodeB.getWidth()
        hlowA, huppA = nodeA.getHeights()
        hlowB, huppB = nodeB.getHeights()

        super(ConcatNode, self).__init__(wid, max(hlowA, hlowB), max(huppA, huppB))
        self.setPosition(nodeA.getPosition())   #Mueve el nodo
        self.addNode(nodeA)                     #Agrega el 1er nodo
        self.addNode(nodeB)                     #Agrega el 2do nodo

        nodeB.setPosition(nodeA.getPosition())  #Mueve el nodo y B a la pos. de A
        nodeB.movePosition(nodeA.getWidth(), 0) #Desplaza B.x por A.width()

class ParenthesisNode(Node):
    def __init__(self, subNode):
        wid = subNode.getWidth() + PARENTHESIS_SPACING * 2
        hlow, hupp = subNode.getHeights()

        super(ParenthesisNode, self).__init__(wid, hlow, hupp)
        self.setPosition(subNode.getPosition())
        subNode.movePosition(PARENTHESIS_SPACING, 0)
        self.addNode(subNode)

    def toSvg(self):
        x, y = self.getPosition()
        scale = self.getScale()
        height = self.getHeight() / .74
        subNodeWidth = self.getWidth() - PARENTHESIS_SPACING

        y += self.getLowerHeight()
        y -= (0.12 * FONT_SIZE) * height / scale

        parStr = "<text x='0' y='0' font-size='%.2f' transform='translate(%.2f, %.2f) 
        		 scale(1, %.2f)'>%s</text>\n"
        lPar = parStr % (scale, x, y, height / scale, '(')
        rPar = parStr % (scale, x + subNodeWidth, y, height / scale, ')')

        return lPar + super(ParenthesisNode, self).toSvg() + rPar

class SubIndexNode(Node):
    def __init__(self, rootNode, indexNode):
        indexNode.scaleBy(0.7)

        wid = rootNode.getWidth() + indexNode.getWidth()
        hlow, hupp = rootNode.getHeights()

        hlow += indexNode.getHeight() + SUB_SPACING
        super(SubIndexNode, self).__init__(wid, hlow, hupp)


        #Mueve el nodo Index a la derecha y un poco mas abajo que Root
        indexNode.movePosition(rootNode.getWidth(), indexNode.getUpperHeight() + 
        			rootNode.getLowerHeight() + SUB_SPACING)
        #Agrega el nodo Index
        self.addNode(indexNode)
        #Setea el nodo en la posicion de Root                             
        self.setPosition(rootNode.getPosition())
        #Agrega el nodo Root            
        self.addNode(rootNode)                              

class SuperIndexNode(Node):
    def __init__(self, rootNode, indexNode):
        indexNode.scaleBy(0.7)

        wid = rootNode.getWidth() + indexNode.getWidth()
        hlow, hupp = rootNode.getHeights()

        hupp += indexNode.getHeight() + SUPER_SPACING
        super(SuperIndexNode, self).__init__(wid, hlow, hupp)
        
        #Mueve el nodo Index a la derecha y un poco mas arriba que Root
        indexNode.movePosition(rootNode.getWidth(), -indexNode.getLowerHeight() - 
                  rootNode.getUpperHeight() - SUPER_SPACING)
        #Agrega el nodo Index
        self.addNode(indexNode)
        #Setea el nodo en la posicion de Root                             
        self.setPosition(rootNode.getPosition())
        #Agrega el nodo Root            
        self.addNode(rootNode)                              

class SuperSubIndexNode(Node):
    def __init__(self, rootNode, superNode, subNode):
        superNode.scaleBy(0.7)
        subNode.scaleBy(0.7)

        wid = rootNode.getWidth() + max(superNode.getWidth(), subNode.getWidth())
        hlow, hupp = rootNode.getHeights()
        hlow += subNode.getHeight() + SUB_SPACING
        hupp += superNode.getHeight() + SUPER_SPACING
        super(SuperSubIndexNode, self).__init__(wid, hlow, hupp)

        #Mueve el Sub a la derecha y abajo del Root
        subNode.movePosition(rootNode.getWidth(), subNode.getUpperHeight() + 
                rootNode.getLowerHeight() + SUB_SPACING)
        
        #Mueve el Super a la derecha y arriba del Root
        superNode.movePosition(rootNode.getWidth(), -superNode.getLowerHeight() - 
                  rootNode.getUpperHeight() - SUPER_SPACING)

		#Agrega el nodo Sub
        self.addNode(subNode)
        #Agrega el nodo Super                               
        self.addNode(superNode)
        #Setea el nodo en la posicion de Root                             
        self.setPosition(rootNode.getPosition())            
        self.addNode(rootNode)  

class DivideNode(Node):
    def __init__(self, upperNode, lowerNode):
        uwid = upperNode.getWidth()
        lwid = lowerNode.getWidth()
        if uwid > lwid:
            wid = uwid
        else:
            wid = lwid
        hlow = lowerNode.getHeight() + CHAR_HEIGHT * 0.28
        hupp = upperNode.getHeight() + CHAR_HEIGHT * 0.4

        lineNode = LineNode(wid)


        super(DivideNode, self).__init__(wid, hlow, hupp)
        #Mueve el Nodo a la posicion de Upper
        self.setPosition(upperNode.getPosition())               
        self.addNode(upperNode)
        self.addNode(lineNode)
        self.addNode(lowerNode)

		#Mueve la linea al Baseline
        lineNode.setPosition(upperNode.getPosition())
        #Mueve la linea un poco arriba para que quede alineada con los '-''               
        lineNode.movePosition(0, -.28*CHAR_HEIGHT)
        #Mueve Lower al Baseline                  
        lowerNode.setPosition(upperNode.getPosition())
        #Mueve Lower por debajo de la linea              
        lowerNode.movePosition(0, lowerNode.getUpperHeight())
        #Mueve Upper por encima de la linea       
        upperNode.movePosition(0, -upperNode.getLowerHeight() -.4*CHAR_HEIGHT) 

        if uwid > lwid:
            lowerNode.movePosition((uwid - lwid) / 2, 0)
        else:
            upperNode.movePosition((lwid - uwid) / 2, 0)


class LineNode(Node):
    def __init__(self, wid):
        super(LineNode, self).__init__(wid, LINE_SPACING, LINE_SPACING)

    def toSvg(self):
        x, y = self.getPosition()
        wid = self.getWidth()
        lineStr = "<line x1='%.2f' y1='%.2f' x2='%.2f' y2='%.2f' stroke-width='%.3f' 
        		   stroke='black' />\n"
        stroke = 0.03 * self.scale
        return (lineStr % (x, y, x + wid, y, stroke)) + super(LineNode, self).toSvg()
\end{verbatim}

