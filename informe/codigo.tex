\section{C\'odigo}

\par En \textbf{main.py} usando \textit{ply} tenemos definidos el \textit{lexer} y el \textit{parser}, una funcion encargada de los tests y el programa propiamente dicho. El Lexer no merece mucha atenci\'on, as\'i que pasaremos directamente al parser:
\par Como ply utiliza LALR y \'este s\'olo permite atributos sintetizados, decidimos tener un \'unico atributo que va formando un \'arbol a medida que se hacen las reducciones. Dicho \'arbol no refleja la sintaxis, sino que es m\'as bien una representaci\'on de las posiciones y tama\~nos de los caracteres de la f\'ormula, junto con su dependencia.
\par En \textbf{Node.py} est\'a el c\'odigo de los Nodos de este \'arbol.
\par Tomando como ejemplo la producci\'on $CONCAT_1 \rightarrow CONCAT_2$ $ELEMENTS$, donde en ply se separan los terminales y no-terminales como $p[0]$ $(CONCAT_1)$, $p[1]$ $(CONCAT_2)$ y $p[2]$ $(ELEMENTS)$, lo que hacemos es $p[0] = ConcatNode(p[1], p[2])$. Es decir, representamos la concatenaci\'on de los dos no-terminales ya procesados como \'arboles en $p[1]$ y $p[2]$.
\par Por otro lado, en $GROUP \rightarrow \{ E \}$, tenemos $p[0]$ $(GROUP)$, $p[1]$ $(\{)$, $p[2]$ $(E)$ y $p[3]$ $(\})$, y en este caso pasamos directamente el \'arbol procesado en $p[2]$. Es decir, $p[0] = p[2]$, ya que no hay ning\'un cambio respecto a la posicion o tama\~no.
\par Estos dos ejemplos muestran que el \'arbol que generamos tiende a ser sint\'actico pero no lo es necesariamente. De hecho, definimos un \texttt{LineNode} que representa la l\'inea de divisi\'on, y que se instancia en la implementaci\'on de \texttt{DivideNode}.
\par Finalmente, en la producci\'on $S \rightarrow E$ simplemente completamos el \'arbol con la ra\'iz y llamamos al m\'etodo $toSvg()$ de \'esta. $p[0] = MainNode(p[1]).toSvg()$, donde $MainNode$ representa la ra\'iz del \'arbol y $toSvg()$ va generando y concatenando los $toSvg$ de los subnodos para obtener el svg final de la f\'ormula.

\par \phantom{a}
\par A continuaci\'on inclu\'imos el c\'odigo de \textbf{main.py} y \textbf{Node.py}:

\subsection{main.py}
\lstinputlisting{../src/main.py}

\newpage

\subsection{Node.py}
\lstinputlisting{../src/Node.py}