\section{Gramática}
La gramática se nos fue presentada fue la siguiente:\\

\begin{tabular}{ l c l }
    E & $\rightarrow$ & EE \\
    & $|$ & E$\wedge$E \\
    & $|$ & E\_E \\
    & $|$ & E$\wedge$E\_E \\
    & $|$ & E\_E$\wedge$E \\
    & $|$ & E\/E \\
    & $|$ & (E) \\
    & $|$ & \{E\} \\
    & $|$ & l \\
\end{tabular}

Dado que dicha gramática tenia problemas de ambiguedad, los cuales preferimos evitar y que había ciertas restricciones del lenguaje que no estaban contempladas decidimos modificarla.
A continuación presentamos la gramática y la explicación de porque representa el mismo lenguaje: \\

Ahora el símbolo distiguido es $S$ el cuál deriba en $E$. Uno de los problemas que teníamos con la ambiguedad es que $E$ podía
ser cualquier tipo de expresión con lo cual decidimos darle una cierta jerarquía a cada expresión (aprovechando que la gramática lo solicitaba) y de este modo
deshacernos de la ambiguedad.\\


\begin{tabular}{ l c l }
    S &        $\rightarrow$ & E \\
    E &        $\rightarrow$ & E \/ CONCAT $|$ CONCAT \\
    CONCAT &   $\rightarrow$ & CONCAT ELEMENTS $|$ ELEMENTS \\
    ELEMENTS & $\rightarrow$ & INDEXES $|$ NOINDEX \\
\end{tabular} \\
$E$ puede convertirse en divisiones o bien pasar a ser una $CONCAT$enación recursiva de $ELEMENTS$ donde cada elemento de la concatenación
puede ser una expresión que contiene $INDEXES$ o bien $NOINDEXES$.\\

$INDEXES$, como su nombre lo indica, contiene todas las expresiones que tienen índices y cada expresión no puede
estar formada por expresiones $INDEXES$ ya que los indices no son asociativos(por enunciado) y porque la concatenación
de subíndices con superíndices genera ambiguedad con la expresión original E $\rightarrow$ E\_E$\wedge$E. \\


\begin{tabular}{ l c l }
    INDEXES &  $\rightarrow$ & SUPER $|$ SUB $|$ SUPSUB $|$ SUBSUP \\
    SUPER &    $\rightarrow$ & NOINDEX$\wedge$NOINDEX \\
    SUB &      $\rightarrow$ & NOINDEX\_NOINDEX \\
    SUPSUB &   $\rightarrow$ & NOINDEX$\wedge$NOINDEX\_NOINDEX \\
    SUBSUP &   $\rightarrow$ & NOINDEX\_NOINDEX$\wedge$NOINDEX \\
\end{tabular}\\

$NOINDEX$ son todas aquellas expresiones que no son formadas por índices: $PAR$, $GROUP$ y $ID$. \\


\begin{tabular}{ l c l }
    NOINDEX &  $\rightarrow$ & PAR $|$ GROUP $|$ ID \\    
    PAR &      $\rightarrow$ & (E) \\
    GROUP &    $\rightarrow$ & \{E\} \\
    ID &       $\rightarrow$ & l \\
\end{tabular} \\
