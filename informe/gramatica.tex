\section{Gramática}
Las producciones de la gram\'atica que se nos fue presentada son de la forma:\\

\begin{tabular}{ l c l }
    E & $\rightarrow$ & EE \\
    & $|$ & E$\wedge$E \\
    & $|$ & E\_E \\
    & $|$ & E$\wedge$E\_E \\
    & $|$ & E\_E$\wedge$E \\
    & $|$ & E$\slash$E \\
    & $|$ & (E) \\
    & $|$ & \{E\} \\
    & $|$ & l \\
\end{tabular}

Como dicha gram\'atica tiene problemas de ambig\"uedad y hay ciertas restricciones del lenguaje que no est\'an contempladas, decidimos modificarla.
A continuaci\'on presentamos nuestra gram\'atica y la explicaci\'on de por qu\'e representa el mismo lenguaje: \\

El s\'imbolo distiguido ahora es $S$, el cual deriva en $E$. Uno de los problemas que ten\'iamos con la ambiguedad es que $E$ pod\'ia ser cualquier tipo de expresi\'on, con lo cual decidimos darle una cierta jerarqu\'ia a cada expresi\'on (aprovechando que la gram\'atica lo solicitaba) y de este modo deshacernos de la ambiguedad.\\


\begin{tabular}{ l c l }
    S &        $\rightarrow$ & E \\
    E &        $\rightarrow$ & E $\slash$ CONCAT $|$ CONCAT \\
    CONCAT &   $\rightarrow$ & CONCAT ELEMENTS $|$ ELEMENTS \\
    ELEMENTS & $\rightarrow$ & INDEXES $|$ NOINDEX \\
\end{tabular} \\
$E$ puede convertirse en divisiones o bien pasar a ser una $CONCAT$enaci\'on recursiva de $ELEMENTS$ donde cada elemento de la concatenaci\'on
puede ser una expresi\'on que contiene $INDEXES$ o bien $NOINDEXES$.\\

$INDEXES$, como su nombre lo indica, contiene todas las expresiones que tienen \'indices. No pueden estar directamente formadas por expresiones $INDEXES$ ya que los indices no son asociativos (por enunciado) y porque la concatenaci\'on
de sub\'indices con super\'indices generar\'ia ambiguedad, por ejemplo con la expresi\'on original E $\rightarrow$ E\_E$\wedge$E. \\


\begin{tabular}{ l c l }
    INDEXES &  $\rightarrow$ & SUPER $|$ SUB $|$ SUPSUB $|$ SUBSUP \\
    SUPER &    $\rightarrow$ & NOINDEX$\wedge$NOINDEX \\
    SUB &      $\rightarrow$ & NOINDEX\_NOINDEX \\
    SUPSUB &   $\rightarrow$ & NOINDEX$\wedge$NOINDEX\_NOINDEX \\
    SUBSUP &   $\rightarrow$ & NOINDEX\_NOINDEX$\wedge$NOINDEX \\
\end{tabular}\\

$NOINDEX$ son todas aquellas expresiones que no son formadas por \'indices: $ID$, $PAR$ y $GROUP$. \\


\begin{tabular}{ l c l }
    NOINDEX &  $\rightarrow$ & PAR $|$ GROUP $|$ ID \\    
    PAR &      $\rightarrow$ & (E) \\
    GROUP &    $\rightarrow$ & \{E\} \\
    ID &       $\rightarrow$ & l \\
\end{tabular} \\
