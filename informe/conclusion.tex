\section{Conclusiones}
Una de las cosas que notamos es que hay gram\'aticas que son ambiguas y eso las hace f\'cilmente comprendibles
por humanos pero las computadoras \textquote{prefieren} gram\'aticas que no lo sean. Aprendimos que con un poco de trabajo
podemos, no solo desambiguar una gram\'tica, sino tambi\'en modificarla para que algunas restricciones que estaban dadas sem\'anticamente
queden abarcadas por la gram\'atica como lo fueron la no asociatividad de los exponentes y la precedencia de algunos operadores. \\

Pudimos ver de una forma pr\'actica como el parser interact\'ua con el lexer y sem\'antica para formar cosas muy poderosas de una
manera muy simple, ya que con peque\~nos ladrillitos como lo son las producciones se generan de forma recursiva cosas inimaginables.
Despu\'es de este trabajo pr\'actico, ver cosas como las que hace \LaTeX nos dejaron de parecer \textquote{magia negra} y pudimos entender
que el concepto por el cual se rigen es el mismo. \\