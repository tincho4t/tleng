\section{Conclusiones}
Una de las cosas que notamos es que hay gramáticas que son ambiguas y eso las hace fácilmente comprendibles
por humanos pero las computadoras \textquote{prefieren} gramáticas que no lo sean. Aprendimos que con un poco de trabajo
podemos, no solo desambiguar una gramática, sino también modificarla para que algunas restricciones que estaban dadas semánticamente
queden abarcadas por la gramática como lo fueron la no asociatividad de los exponentes y la presedencia de algunos operadores. \\

Pudimos ver de una forma práctica como el parser interactúa con el lexer y semántica para formar cosas muy poderosas de una
manera muy simple, ya que con pequeños ladrillitos como lo son las producciones se generan de forma recursiva cosas inimaginables.
Después de este trabajo práctico, ver cosas como las que hace \LaTeX nos dejaron de parecer \textquote{magia negra} y pudimos entender
que el concepto por el cual se rigen es el mismo. \\